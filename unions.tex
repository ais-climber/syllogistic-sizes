\documentclass[12pt]{article}
\usepackage{amssymb,amsthm,amsmath}
\usepackage{lscape}

\usepackage{bigstrut}
%\usepackage{MnSymbol}
\usepackage{bbm}
\usepackage{proof}
\usepackage{bussproofs}
\usepackage{tikz}
\usepackage{lingmacros}


\usepackage{hyperref}
\hypersetup{
    colorlinks,
    citecolor=black,
    filecolor=black,
    linkcolor=black,
    urlcolor=black
}


\newcommand{\existsgeq}{\mbox{\sf AtLeast}}
\newcommand{\Pol}{\mbox{\emph{Pol}}}
  \newcommand{\nonered}{\textcolor{red}{=}}
  \newcommand{\equalsred}{\nonered}
  \newcommand{\redstar}{\textcolor{red}{\star}}
    \newcommand{\dred}{\textcolor{red}{d}}
    \newcommand{\dmark}{\dred}
    \newcommand{\redflip}{\textcolor{red}{flip}}
        \newcommand{\flipdred}{\textcolor{red}{\mbox{\scriptsize \em flip}\ d}}
        \newcommand{\mdred}{\textcolor{blue}{m}\textcolor{red}{d}}
        \newcommand{\ndred}{\textcolor{blue}{n}\textcolor{red}{d}}
\newcommand{\arrowm}{\overset{\textcolor{blue}{m}}{\rightarrow} }
\newcommand{\arrown}{\overset{\textcolor{blue}{n}}{\rightarrow} }
\newcommand{\arrowmn}{\overset{\textcolor{blue}{mn}}{\longrightarrow} }
\newcommand{\arrowmonemtwo}{\overset{\textcolor{blue}{m_1 m_2}}{\longrightarrow} }
\newcommand{\bluen}{\textcolor{blue}{n}}
\newcommand{\bluem}{\textcolor{blue}{m}}
\newcommand{\bluemone}{\textcolor{blue}{m_1}}
\newcommand{\bluemtwo}{\textcolor{blue}{m_2}}
\newcommand{\blueminus}{\textcolor{blue}{-}}

\newcommand{\bluedot}{\textcolor{blue}{\cdot}}
\newcommand{\bluepm}{\textcolor{blue}{\pm}}
\newcommand{\blueplus}{\textcolor{blue}{+ }}
\newcommand{\translate}[1]{{#1}^{tr}}
\newcommand{\Caba}{\mbox{\sf Caba}} 
\newcommand{\Set}{\mbox{\sf Set}} 
\newcommand{\Pre}{\mbox{\sf Pre}} 
\newcommand{\wmarkpolarity}{\scriptsize{\mbox{\sf W}}}
\newcommand{\wmarkmarking}{\scriptsize{\mbox{\sf Mon}}}
\newcommand{\smark}{\scriptsize{\mbox{\sf S}}}
\newcommand{\bmark}{\scriptsize{\mbox{\sf B}}}
\newcommand{\mmark}{\scriptsize{\mbox{\sf M}}}
\newcommand{\jmark}{\scriptsize{\mbox{\sf J}}}
\newcommand{\kmark}{\scriptsize{\mbox{\sf K}}}
\newcommand{\tmark}{\scriptsize{\mbox{\sf T}}}
\newcommand{\greatermark}{\mbox{\tiny $>$}}
\newcommand{\lessermark}{\mbox{\tiny $<$}}
%%{\mbox{\ensuremath{>}}}
\newcommand{\true}{\top}
\newcommand{\false}{\bot}
\newcommand{\upred}{\textcolor{red}{\uparrow}}
\newcommand{\downred}{\textcolor{red}{\downarrow}}
\usepackage[all,cmtip]{xy}
\usepackage{enumitem}
\usepackage{fullpage}
\usepackage[authoryear]{natbib}
\usepackage{multicol}
\theoremstyle{definition}
\newtheorem{definition}{Definition}
\newtheorem{theorem}{Theorem}
\newtheorem{lemma}[theorem]{Lemma}
\newtheorem{claim}{Claim}
\newtheorem{corollary}{Corollary}
%\newtheorem{theorem}{Theorem}
\newtheorem{proposition}{Proposition}
\newtheorem{example}{Example}
\newtheorem{remark}[theorem]{Remark}
\newcommand{\semantics}[1]{[\![\mbox{\em $ #1 $\/}]\!]}
\newcommand{\abovearrow}[1]{\rightarrow\hspace{-.14in}\raiseonebox{1.0ex}
{$\scriptscriptstyle{#1}$}\hspace{.13in}}
\newcommand{\toplus}{\abovearrow{r}}
\newcommand{\tominus}{\abovearrow{i}} 
\newcommand{\todestroy}{\abovearrow{d}}
\newcommand{\tom}{\abovearrow{m}}
\newcommand{\tomprime}{\abovearrow{m'}}
\newcommand{\A}{\textsf{App}}
\newcommand{\At}{\textsf{At}}
\newcommand{\Emb}{\textsf{Emb}}
\newcommand{\EE}{\mathbb{E}}
\newcommand{\DD}{\mathbb{D}}
\newcommand{\PP}{\mathbb{P}}
\newcommand{\QQ}{\mathbb{Q}}
\newcommand{\LL}{\mathbb{L}}
\newcommand{\MM}{\mathbb{M}}
\usepackage{verbatim}
\newcommand{\TT}{\mathcal{T}}
\newcommand{\Marking}{\mbox{Mar}}
\newcommand{\Markings}{\Marking}
\newcommand{\Mar}{\Marking}
\newcommand{\Model}{\mathcal{M}}
\renewcommand{\SS}{\mathcal{S}}
\newcommand{\TTM}{\TT_{\Markings}}
\newcommand{\CC}{\mathbb{C}}
\newcommand{\erase}{\mbox{\textsf{erase}}}
\newcommand{\set}[1]{\{ #1 \}}
\newcommand{\arrowplus}{\overset{\blueplus}{\rightarrow} }
\newcommand{\arrowminus}{\overset{\blueminus}{\rightarrow} }
\newcommand{\arrowdot}{\overset{\bluedot}{\rightarrow} }
\newcommand{\arrowboth}{\overset{\bluepm}{\rightarrow} }
\newcommand{\arrowpm}{\arrowboth}
\newcommand{\arrowplusminus}{\arrowboth}
\newcommand{\arrowmone}{\overset{m_1}{\rightarrow} }
\newcommand{\arrowmtwo}{\overset{m_2}{\rightarrow} }
\newcommand{\arrowmthree}{\overset{m_3}{\rightarrow} }
\newcommand{\arrowmcomplex}{\overset{m_1 \orr m_2}{\longrightarrow} }
\newcommand{\arrowmproduct}{\overset{m_1 \cdot m_2}{\longrightarrow} }
\newcommand{\proves}{\vdash}
\newcommand{\Dual}{\mbox{\sc dual}}
\newcommand{\orr}{\vee}
\newcommand{\uar}{\uparrow}
\newcommand{\dar}{\downarrow}
\newcommand{\andd}{\wedge}
\newcommand{\bigandd}{\bigwedge}
\newcommand{\arrowmprime}{\overset{m'}{\rightarrow} }
\newcommand{\quadiff}{\quad \mbox{ iff } \quad}
\newcommand{\Con}{\mbox{\sf Con}}
\newcommand{\type}{\mbox{\sf type}}
\newcommand{\lang}{\mathcal{L}}
\newcommand{\necc}{\Box}
\newcommand{\vocab}{\mathcal{V}}
\newcommand{\wocab}{\mathcal{W}}
\newcommand{\Types}{\mathcal{T}_\mathcal{M}}
\newcommand{\mon}{\mbox{\sf mon}}
\newcommand{\anti}{\mbox{\sf anti}}
\newcommand{\FF}{\mathcal{F}}
\newcommand{\rem}[1]{\relax}


\newcommand{\raiseone}{\mbox{raise}^1}
\newcommand{\raisetwo}{\mbox{raise}^2}
\newcommand{\wrapper}[1]{{#1}}
\newcommand{\sfa}{\wrapper{\mbox{\sf a}}}
\newcommand{\sfb}{\wrapper{\mbox{\sf b}}}
\newcommand{\sfv}{\wrapper{\mbox{\sf v}}}
\newcommand{\sfw}{\wrapper{\mbox{\sf w}}}
\newcommand{\sfx}{\wrapper{\mbox{\sf x}}}
\newcommand{\sfy}{\wrapper{\mbox{\sf y}}}
\newcommand{\sfz}{\wrapper{\mbox{\sf z}}}
  \newcommand{\sff}{\wrapper{\mbox{\sf f}}}
    \newcommand{\sft}{\wrapper{\mbox{\sf t}}}
      \newcommand{\sfc}{\wrapper{\mbox{\sf c}}}
      \newcommand{\sfu}{\wrapper{\mbox{\sf u}}}
            \newcommand{\sfs}{\wrapper{\mbox{\sf s}}}
  \newcommand{\sfg}{\wrapper{\mbox{\sf g}}}

\newcommand{\sfvomits}{\wrapper{\mbox{\sf vomits}}}
\newlength{\mathfrwidth}
  \setlength{\mathfrwidth}{\textwidth}
  \addtolength{\mathfrwidth}{-2\fboxrule}
  \addtolength{\mathfrwidth}{-2\fboxsep}
\newsavebox{\mathfrbox}
\newenvironment{mathframe}
    {\begin{lrbox}{\mathfrbox}\begin{minipage}{\mathfrwidth}\begin{center}}
    {\end{center}\end{minipage}\end{lrbox}\noindent\fbox{\usebox{\mathfrbox}}}
    \newenvironment{mathframenocenter}
    {\begin{lrbox}{\mathfrbox}\begin{minipage}{\mathfrwidth}}
    {\end{minipage}\end{lrbox}\noindent\fbox{\usebox{\mathfrbox}}} 
 \renewcommand{\hat}{\widehat}
 \newcommand{\nott}{\neg}
  \newcommand{\preorderO}{\mathbb{O}}
 \newcommand{\PreorderP}{\mathbb{P}}
  \newcommand{\preorderE}{\mathbb{E}}
\newcommand{\preorderP}{\mathbb{P}}
\newcommand{\preorderN}{\mathbb{N}}
\newcommand{\preorderQ}{\mathbb{Q}}
\newcommand{\preorderX}{\mathbb{X}}
\newcommand{\preorderA}{\mathbb{A}}
\newcommand{\preorderR}{\mathbb{R}}
\newcommand{\preorderOm}{\mathbb{O}^{\bluem}}
\newcommand{\preorderPm}{\mathbb{P}^{\bluem}}
\newcommand{\preorderQm}{\mathbb{Q}^{\bluem}}
\newcommand{\preorderOn}{\mathbb{O}^{\bluen}}
\newcommand{\preorderPn}{\mathbb{P}^{\bluen}}
\newcommand{\preorderQn}{\mathbb{Q}^{\bluen}}
 \newcommand{\PreorderPop}{\mathbb{P}^{\blueminus}}
  \newcommand{\preorderEop}{\mathbb{E}^{\blueminus}}
\newcommand{\preorderPop}{\mathbb{P}^{\blueminus}}
\newcommand{\preorderNop}{\mathbb{N}^{\blueminus}}
\newcommand{\preorderQop}{\mathbb{Q}^{\blueminus}}
\newcommand{\preorderXop}{\mathbb{X}^{\blueminus}}
\newcommand{\preorderAop}{\mathbb{A}^{\blueminus}}
\newcommand{\preorderRop}{\mathbb{R}^{\blueminus}}
 \newcommand{\PreorderPflat}{\mathbb{P}^{\flat}}
  \newcommand{\preorderEflat}{\mathbb{E}^{\flat}}
\newcommand{\preorderPflat}{\mathbb{P}^{\flat}}
\newcommand{\preorderNflat}{\mathbb{N}^{\flat}}
\newcommand{\preorderQflat}{\mathbb{Q}^{\flat}}
\newcommand{\preorderXflat}{\mathbb{X}^{\flat}}
\newcommand{\preorderAflat}{\mathbb{A}^{\flat}}
\newcommand{\preorderRflat}{\mathbb{R}^{\flat}}
\newcommand{\pstar}{\preorderBool^{\preorderBool^{E}}}
\newcommand{\pstarplus}{(\pstar)^{\blueplus}}
\newcommand{\pstarminus}{(\pstar)^{\blueminus}}
\newcommand{\pstarm}{(\pstar)^{\bluem}}
\newcommand{\Reals}{\preorderR}
\newcommand{\preorderS}{\mathbb{S}}
\newcommand{\preorderBool}{\mathbbm{2}}
 \renewcommand{\o}{\cdot}
 \newcommand{\NP}{\mbox{\sc np}}
 \newcommand{\NPplus}{\NP^{\blueplus}}
  \newcommand{\NPminus}{\NP^{\blueminus}}
   \newcommand{\NPplain}{\NP}
    \newcommand{\npplus}{np^{\blueplus}}
  \newcommand{\npminus}{np^{\blueminus}}
   \newcommand{\npplain}{np}
   \newcommand{\np}{np}
   \newcommand{\Term}{\mbox{\sc t}}
  \newcommand{\N}{\mbox{\sc n}}
   \newcommand{\X}{\mbox{\sc x}}
      \newcommand{\Y}{\mbox{\sc y}}
            \newcommand{\V}{\mbox{\sc v}}
    \newcommand{\Nbar}{\overline{\mbox{\sc n}}}
    \newcommand{\Pow}{\mathcal{P}}
    \newcommand{\powcontravariant}{\mathcal{Q}}
    \newcommand{\Id}{\mbox{Id}}
    \newcommand{\pow}{\Pow}
   \newcommand{\Sent}{\mbox{\sc s}}
   \newcommand{\lookright}{\slash}
   \newcommand{\lookleft}{\backslash}
   \newcommand{\dettype}{(e \to t)\arrowminus ((e\to t)\arrowplus t)}
\newcommand{\ntype}{e \to t}
\newcommand{\etttype}{(e\to t)\arrowplus t}
\newcommand{\nptype}{(e\to t)\arrowplus t}
\newcommand{\verbtype}{TV}
\newcommand{\who}{\infer{(\nptype)\arrowplus ((\ntype)\arrowplus (\ntype))}{\mbox{who}}}
\newcommand{\iverbtype}{IV}
\newcommand{\Nprop}{\N_{\mbox{prop}}}
\newcommand{\VP}{{\mbox{\sc vp}}}
\newcommand{\CN}{{\mbox{\sc cn}}}
\newcommand{\Vintrans}{\mbox{\sc iv}}
\newcommand{\Vtrans}{\mbox{\sc tv}}
\newcommand{\Num}{\mbox{\sc num}}
%\newcommand{\S}{\mathbb{A}}
\newcommand{\Det}{\mbox{\sc det}}
\newcommand{\preorderB}{\mathbb{B}}
\newcommand{\simA}{\sim_A}
\newcommand{\simB}{\sim_B}
\newcommand{\polarizedtype}{\mbox{\sf poltype}}
\newcommand{\card}{\mbox{card}}

\begin{document}
%\tableofcontents


\section{The logic of {\sf All} and set unions}

Here's the syntax.   We start with \emph{basic nouns} and from these we construct \emph{union terms}
We
use letters $x$, $y$, $z$, for basic nouns.  The union terms are terms $x\cup y$, where $x$ and $y$ are basic nouns.
We use letters like $t$ for terms which are either basic nouns  or union terms.

In the semantics, we interpret the basic noun $x$ by $\semantics{x}\subseteq M$, and then we always interpret a union term $x\cup y$ by 
$\semantics{x}\cup\semantics{y}$.

\begin{figure}[t]
\begin{mathframe}
\[
\begin{array}{l@{\qquad}l@{\qquad}l}
\infer{\mbox{\sf All $t$ $t$}}{}
&
\infer{\mbox{\sf All $t$ $v$}}{\mbox{\sf All $t$ $u$} & \mbox{\sf All $u$ $v$}}
&
\infer{\mbox{\sf All ($x\cup x$) $x$}}{}  \\  \\
\infer{\mbox{\sf All $x$ ($x\cup y$) }}{} &
\infer{\mbox{\sf All ($y \cup x$) ($x\cup y$) }}{} &
\infer{\mbox{\sf All ($x\cup y$) $t$}}{\mbox{\sf All $x$ $t$} & \mbox{\sf All $y$ $t$}}
\end{array}
\]
\caption{The logic of {\sf All} and set unions.\label{fig-all-unions}}
\end{mathframe}
\end{figure}

For a fixed set $\Gamma$, we write $t\leq u$ to mean that $\Gamma\proves \mbox{\sf All $t$ $u$}$.
(We do this to lighten the notation.)    We also write $x \equiv y$ to mean $x\leq y \leq x$.

\begin{example}
For any set $\Gamma$, if $x\leq y$ and $z\leq w$, then $x\cup z \leq y\cup w$.
\label{ex-1}
\end{example}

\begin{example}
For any set $\Gamma$, if $a\equiv x\cup y$, $b\equiv a\cup z$, 
$c \equiv y \cup z$, and $d \equiv x \cup c$, then $b \equiv d$.
\label{ex-2}
\end{example}

\begin{definition} 
A set $S$ of terms is an \emph{up-set (for $\Gamma$)} if whenever $t\in S$ and $t\leq u$, then also $u\in S$.
$S$ is \emph{prime} if whenever $x\cup y \in S$, then either $x\in S$ or $y\in S$.
\end{definition}

Note that the notion of an up-set is relative to a set $\Gamma$, but the notion of a prime set does not refer to any set at all.

When $\Gamma$ is clear from the context, we just speak of a set $S$ being an up-set (without referencing $\Gamma$).

\begin{example}
If $\Model$ is any model, then for all $m\in M$, $S_m = \set{t : m \in \semantics{t}}$
is   prime.  If $\Model\models\Gamma$, then $S_m$ is an up-set for $\Gamma$.
\label{ex-3}
\end{example}

\begin{lemma}  Fix a set $\Gamma$.
Let $t$ be any term, and assume that $t \not\leq y\cup z$.
Then there is a prime up-set containing $t$ but not containing either $y$ or $z$.
\label{lemma-zorn}
\end{lemma}

\begin{proof}

Let $\SS$ be the family of sets $T$ which contains $t$, is closed upwards, and contains neither  $y$ or $z$.
One such set in $\SS$ is $\uparrow t$.  Note first that $\uparrow t$ does not contain either $y$ or $z$.  (For if $t\leq y$, then since $y\leq y \cup z$, we would have a contradiction.)

By Zorn's Lemma, let $S$ be a maximal element of $\SS$ with respect to inclusion.
We claim that 
$S$ is  prime.   To see this, suppose that $a \cup b\in S$.  Suppose towards a contradiction that neither $a$ nor $b$ were in $S$.
By maximality, $S\cup\uparrow a$ and $S\cup\uparrow b$  would not belong to $\SS$. 
So they each contain $y$ or $z$.   Without loss of generality, $a\leq y$ and $b\leq z$.  
By Example~\ref{ex-1},  $a\cup b \leq y\cup z$.   Since $S$ is an up-set, $y\cup z$ belongs to $S$.    And this is a contradiction.
\end{proof}

\begin{theorem}[Completeness]
The logic of {\sf all} and unions in Figure~\ref{fig-all-unions} is complete.
\label{theorem-first-completeness-union}
\end{theorem}

\begin{proof}
We need to show that if $\Gamma\models \mbox{\sf All $t$ $u$}$,
$\Gamma\proves \mbox{\sf All $t$ $u$}$.
We may assume that $u$ is a union term.  (If $u$ were a basic noun $x$, replace $x$ with $x\cup x$.)
We also may assume that $t$ is a basic noun.   Here is the reason.   Suppose that our original assumption were
$\Gamma\models \mbox{\sf All ($x\cup y$) $u$}$.   It follows that both $\Gamma\models \mbox{\sf All $x$ $u$}$
and $\Gamma\models \mbox{\sf All $y$ $u$}$.   If we were to prove that $x \leq u$ and $y\leq u$, then by the logic,
we would have our desired conclusion:
$x\cup y \leq y$.

Thus, we reduce to showing that if  $\Gamma\models x\leq y \cup z$, then also  $\Gamma \proves x\leq y \cup z$.
We show the contrapositive.   Assume
 that $\Gamma\not\proves x\leq y \cup z$.   We shall find a model of $\Gamma$ where
$ \semantics{x} \not\subseteq (\semantics{y}\cup\semantics{z})$.
By Lemma~\ref{lemma-zorn}, let $S$ be a prime up-set containing $x$ but not containing either $y$ or $z$.

We use $S$ to make a model $\Model$ with one point, say $*$.   We put $*\in \semantics{u}$ iff $u\in S$.
Let us check that $\Model\models \Gamma$.  
Suppose that $\Gamma$ contains the sentence {\sf All $ a$ $(b\cup c)$.}    We may assume that $\semantics{a} = \set{*}$, 
since otherwise $\semantics{a} = \emptyset$, and trivially $\semantics{a}\subseteq \semantics{b}\cup\semantics{c}$.
So $a \in S$.  As $S$ is closed upwards and $a\leq b\cup c$, $b\cup c\in S$ also.   Since $S$ is prime, either $b\in S$ or $c\in S$.
So either $*\in\semantics{b}$ or $*\in \semantics{c}$.  Either way, $\semantics{b}\cup\semantics{c} = \set{*}$.  And again we have 
$\semantics{a}\subseteq \semantics{b}\cup\semantics{c}$.
Thus, $\Model\models \Gamma$.  

By the defining property of $S$, $*\in \semantics{x} \setminus (\semantics{y}\cup\semantics{z})$ in our model.   Thus, 
$ \semantics{x} \not\subseteq (\semantics{y}\cup\semantics{z})$.
So we are done.
\end{proof}

\subsection{Constructing models from prime up-sets}

Here is a more general result than what we saw in Theorem~\ref{theorem-first-completeness-union}.

\begin{lemma} 
Fix $\Gamma$.
Let $M$ be any set, and suppose that for each $m\in M$ we have a prime up-set of terms $T_m$. 
Equip $M$ with the structure of a model by  interpreting 
basic nouns on $M$ thus:
 \[
 \semantics{x} = \set{m\in M : x \in T_m }.
 \]
 Then $\Model\models\Gamma$.   Moreover, for all terms $t$, 
  \begin{equation}
  \semantics{t} = \set{m\in M : t \in T_m }.
  \label{mono}
    \end{equation}
 \end{lemma}
 
 \begin{proof}
Since each $T_m$ is prime, 
 \[
 \semantics{x\cup y} =  \set{m\in M : x \in T_m }\cup\set{m\in M : y \in T_m } = \set{m\in M : x\cup y  \in T_m }.
 \]
 (\ref{mono}) follows, for all terms $t$.
 We are left with the verification that $\Model\models\Gamma$.
 Suppose that $\Gamma$ contains the sentence {\sf All $u$ $t$}.  
 Then $u \leq t$.
 Let $m\in \semantics{u}$, so by (\ref{mono}), $u\in T_m$.
 Since $T_m$ is an up-set, $t\in T_m$.    By (\ref{mono}) again, $m\in \semantics{t}$.  This shows that 
$\semantics{u} \subseteq \semantics{t}$; that is, our sentence  {\sf All $u$ $t$}  is true in $\Model$. 
 \end{proof}
 
 Another fact worth knowing: the union of two prime up-sets is again a prime up-set.
 

 \section{Adding {\sf Some $t$ $u$}}
 
 \section{Adding {\sf More $t$ $u$} and $\existsgeq(x,y)$}
 
 
 \begin{figure}[t]
\begin{mathframe}
 \[
 \infer{\mbox{\sf Some} (a, c)}{\mbox{\sf More}(a,b) & \mbox{\sf AtLeast}(c,d) & \mbox{\sf{AtLeast}}(b \cup d, a \cup c)}
\] 
 
 
\caption{The Friday rule.\label{fig-friday}}
\end{mathframe}
\end{figure}
 

\vfil\eject
 \section{A lemma on strict orders}
 
Let $I = \set{1,\ldots, n}$, and write $J$ for the set of unordered pairs from $I$.
$J$ consists of all sets $\set{i,j}$ with $i\neq j$ taken from $I$.  If $\set{i,j} \neq \set{k,\ell}$ in $J$,
then either $i \notin \set{k,\ell}$ or $j\notin \set{k,\ell}$;  and also either
$k\notin \set{i,j}$ or else $\ell\notin{i,j}$.

\begin{definition}
A \emph{family over $I$} is a 
function $A$ whose domain is $I$ 
and such that for each $x\in I$, $A_x$ is a  finite set.
\end{definition}
 
\begin{proposition}
There is a family $A$  over $I$ such that whenever
$\set{i,j} \neq \set{k,\ell} \in J$,  $ \card(A_i \cup A_j) \neq  \card(A_k \cup A_{\ell})$.
 % \[  | \card(A_i \cup A_j)-  \card(A_k \cup A_{\ell}) \ | \geq 1  \]
  \label{proposition-first}
  \end{proposition}
 
 \begin{proof}
 Let $A_1 = \set{1}$,
 $A_2 = \set{2,3}$, 
  $A_3 = \set{4, 5, 6, 7}$, 
  $\ldots$, 
 %$A_3 = \set{7, \ldots, 14}$, 
 $A_i = \set{2^{i-1}, \ldots, 2^{i} -1}$, $\ldots$.
 Then $\card(A_i \cup A_j) = 2^{i-1} + 2^{j-1}$.  
 Our result then follows from the uniqueness of binary representations of natural numbers.
   \end{proof}
   
\rem{ Note that $A_1\cup A_3 = \set{1,2,3}$ and $A_2 \cup A_3 = \set{1,4,5,6,7}$, and $5 - 3 =2$.
 In all other cases,   $ | \card(A_i \cup A_j)-  \card(A_k \cup A_{\ell}) \ | > 2 $.
  }

  
   \begin{lemma}
 Let $<$ be a strict linear order of $J$.  Then there is a family $A$ over $I$ such that
whenever $\set{i,j} \neq \set{k,\ell} \in J$, 
 \[ \set{i,j} < \set{k,\ell} 
 \quadiff
 |A_i \cup A_j | < |A_k \cup A_{\ell}|.
 \]
 \label{lemma-J}
 \end{lemma}
 
\begin{proof} 
Fix a family   $A$ over $I$ as in Proposition~\ref{proposition-first}.
Let $\prec$ be the linear order of $J$  defined by
$\set{i,j} \prec \set{k,\ell} $  iff    $ \card(A_i \cup A_j) <  \card(A_k \cup A_{\ell})$.
The statement of Proposition~\ref{proposition-first} implies that $\prec$ is
a linear order.
In general, $< $ and $\prec$ are different
strict linear orders of $J$.
By following the bubble sort algorithm, there is a sequence of strict linear orders of $J$,
\[
\begin{array}{cccccccc}
\prec = <_0 ,
& <_1 ,
& <_2 ,
& \cdots
& <_s,
& \cdots
& <_r = < 
\end{array}
\]
where $r \leq {n+1 \choose 2}$, and where each  $<_s$ and its successor $<_{s+1}$ differ by only a transposition
of adjacent elements.
That is, the two orders  $<_{s}$ and  $<_{s+1}$ are the same except that two elements of 
$J$, say $\set{i,j}$ and $\set{k,\ell}$,  have the property that 
\begin{equation}
\label{ijkl}
\begin{array}{l}
\mbox{$\set{k,\ell}$ is the immediate successor of  $\set{i,j}$ in $<_s$, and  $ \set{k,\ell} <_{s+1} \set{i,j}$.}\\
\mbox{Also, $\set{i,j}$ and $\set{k,\ell}$ are the only pair where $<_s$ and $<_{s+1}$ differ.}
\end{array}
\end{equation}
We show by induction on $s = 0, \ldots, m$ that there is a family $A^s$ such that 
for $\set{a,b} \neq \set{c,d}$ in $J$,
\begin{equation}\label{lemmagoal}
 \set{a,b} <_s \set{c,d} 
 \quadiff
\card(A^{s}_a \cup A^{s}_b ) < \card(A^{s}_c \cup A^{s}_{d}).
 \end{equation}

For $s= 0$, we take the family $A^{0}$ to be  the same family $A$ 
that we fixed at the outset of this proof.  Our choice of $<_0$ as $\prec$ insures that (\ref{lemmagoal}) holds for $s =0$.

Given the family $A^{s}$ for a fixed $s$, we want to define  the family $A^{s+1}$.
To save on superscripts, we'll abbreviate $A^{s}$ by $A$.   And we'll define a new family $B$ and show that it satisfies the 
statement of (\ref{lemmagoal}) for $s+1$ with $B$ replacing $A^{s+1}$.





Fix $i^*$, $j^*$, $k^*$, and $\ell^*$ such that (\ref{ijkl}) holds.
Let $\set{g^*,h^*}$ be the immediate predecessor of $\set{i^*,j^*}$ in $<_s$.
At this point, we need to take a step to insure 
that our family ${A}_x$ has the following extra property:

\begin{equation}\label{lemmagoal2}
    \card({A}_{i^*} \cup {A}_{j^*}) - \card({A}_{g^*} \cup {A}_{h^*})   \geq 2. 
\end{equation}
By (\ref{lemmagoal}) for $s$,   (\ref{lemmagoal2}) holds with the $2$ on the right replaced by $1$.
If need be, take two copies of all the points in all the sets.  This
change doubles the cardinalities of all our sets and all of their unions.
So it preserves  (\ref{lemmagoal}), and it
 will guarantee that (\ref{lemmagoal2}) holds.
 
 
Let \begin{equation}\label{Meq}
 M = \card({A}_{k^*} \cup {A}_{\ell^*}) - \card({A}_{i^*} \cup {A}_{j^*}).\end{equation}
Note that $M > 0$, by (\ref{lemmagoal}).
Take $M+ 1$ fresh points, say $p_1, \ldots, p_{M+1}$, and add them to every set other than 
${A}_{k^*}$ and ${A}_{\ell^*}$.   
In other words, for $x\in I$,
\[
B_{x} = \left\{
\begin{array}{ll}
{A}_x \cup \set{p_1, \ldots, p_{M+1}}  & \mbox{if $x \neq k^*$ and $x \neq \ell^*$} \\
{A}_x   & \mbox{if $x = k^*$ or $x = \ell^*$} \\
\end{array}
\right.
\]
Let us check that (\ref{lemmagoal}) holds for $s+1$.
For all $\set{a,b} \neq \set{k^*, \l^*}$,  either $a$ or $b$ (or both) is not in $\set{k^*, \l^*}$.  And so in this case
the new points will belong to $B_{a} \cup B_b$.  So
\begin{equation}\label{andso}
 \card(B_a \cup  B_b)   = \card({A}_a \cup  {A}_b) + (M + 1) \end{equation}
In the other case,  $\set{a,b} = \set{k^*, \l^*}$,   and $B_{k^*} \cup  B_{\ell^*}  = {A}_{k^*}  \cup  {A}_{\ell^*} $.  So
 \begin{equation}\label{defh}
 \card(B_{k^*} \cup  B_{\ell^*} )   = \card({A}_{k^*}  \cup  {A}_{\ell^*} )  \end{equation}
Now take $\set{a,b}$ and $\set{c,d}$ in $J$.
If neither $\set{a,b}$ nor $\set{c,d}$ is $\set{k^*,\ell^*}$, then 
\[
\begin{array}{cll}
 & \set{a,b} <_{s+1} \set{c,d} \\
 \mbox{iff} &  \set{a,b} <_{s} \set{c,d} & \mbox{by (\ref{ijkl})}\\
  \mbox{iff} & \card({A}_i \cup {A}_j ) < \card({A}_k \cup {A}_{\ell}) &\mbox{by  (\ref{lemmagoal}) for $s$} \\
  \mbox{iff} & \card({A}_i \cup {A}_j ) + (M+1)  < \card({A}_k \cup {A}_{\ell})  + (M+1) \\ 
 \mbox{iff} & \card(B_i \cup B_j )  < \card(B_k \cup B_{\ell}) &\mbox{by (\ref{andso})}    \\ 
\end{array}
\]
From now on in the  verification of (\ref{lemmagoal}), we assume that $\set{c,d}$ is $\set{k^*,\ell^*}$.
As for $\set{a,b}$, we have three alternatives:
 $\set{a,b}= \set{i^*,j^*}$, or $\set{a,b} <_s  \set{i^*,j^*}$, or $ \set{k^*,\ell^*} <  \set{a,b}$.
In the first case,   $\set{k^*,\ell^*} <_{s+1} \set{i^*,j^*}$.   And as desired:
\[
\begin{array}{cll}
& \card(B_{k^*} \cup  B_{\ell^*})\\
=  &  \card({A}_{k^*} \cup  {A}_{\ell^*})  &\mbox{by (\ref{defh})}\\
 <  & \card({A}_{i^*} \cup  {A}_{j^*}) 
+ (M+1) &\mbox{by (\ref{Meq})} \\
=  & \card(B_{i^*} \cup  B_{j^*}) 
&\mbox{by (\ref{andso})}
\\
\end{array}
\]

\rem{

%%%%
  & \card(B_{i^*} \cup  B_{j^*})\\
 =  & \card({A}_{i^*} \cup  {A}_{j^*}) 
+ (M+1) \\
> &  \card({A}_{k^*} \cup  {A}_{\ell^*}) \\
 = & \card(B_{k^*} \cup  B_{\ell^*})
 }
If $\set{a,b} <_s \set{i^*, j^*}$, then also $\set{a,b} <_{s+1} \set{k^*, \ell^*}$,
since $<_s$ and $<_{s+1}$ only differ on their ordering of $\set{i^*, j^*}$ and $\set{k^*, \ell^*}$.
%And in this case, we also have
%\begin{equation}\label{alsohave}
%   \card({A}_{i^*} \cup  {A}_{j^*}) -  \card({A}_{a} \cup  {A}_{b}) \geq 2 \end{equation}
We have the desired result:
\[
\begin{array}{cll}
  & \card(B_{a} \cup  B_{b})\\
  = & \card({A}_{a} \cup  {A}_{b}) + (M+1) &\mbox{by (\ref{andso})} \\
\leq
 & \card({A}_{g^*} \cup  {A}_{h^*}) + (M+1) & \mbox{since $\set{a,b} \leq_s \set{g^*,h^*}$} \\
 \leq &  \card({A}_{i^*} \cup  {A}_{j^*}) + (M+1) -2   & \mbox{by (\ref{lemmagoal2})} \\
<  &  \card({A}_{i^*} \cup  {A}_{j^*}) + M & \\
 =   & \card(B_{k^*} \cup  B_{\ell^*}) &  \mbox{by (\ref{Meq})}\\  
\end{array}
\]  
  
Concluding the verification of (\ref{lemmagoal}) for $s+1$,
 if  $\set{k^*, \ell^*} <_s \set{a,b}$, then  as before  we have $\set{k^*, \ell^*} <_{s+1} \set{a,b}$.
And also 
\[
\begin{array}{cll}
  & \card(B_{k^*} \cup  B_{\ell^*})\\
 =  & \card({A}_{k^*} \cup  {A}_{\ell^*}) 
   & \mbox{by (\ref{defh})}\\
<  &  \card({A}_{a} \cup  {A}_{b})   & \mbox{by (\ref{lemmagoal}) for $s$}  \\
<  &  \card({A}_{a} \cup  {A}_{b}) + (M+1)  & \mbox{since $M>0$}  \\
= & \card(B_{a} \cup  B_{b}) & \mbox{by (\ref{andso})}\\
\end{array}
\]

This completes the proof. \end{proof} 
  
  \end{document}
 
 
 
 
 
 
 
 
 
 
 
 
 
 
 
 
 
 
 
 
 
 
 
 
 








\end{document}